\section{Caracterización y construcción}

\begin{definition}[Distribución inicial]
Al vector de probabilidad $\mu = (\mu_{i})_{i\in I}$ definido por $\mu_{i\in I} = \probf{X_0 = i}$ le llamamos la \textbf{distribución inicial} de la cadena de Markov $X$.
\end{definition}

\begin{proposition}
$X$ es una cadena de Markov con matriz de transición $P$ y distribución inicial $\mu$ si y solo si
\begin{multline}
    \probf{X_0 = i_0,\, X_1 = i_1,\, \ldots,\, X_n = i_n} \\
    = \mu_{i_0} p_{i_0 i_1} \ldots p_{i_{n-1} i_n}
\end{multline}
\end{proposition}

\begin{theorem}[Existencia de una CM con $\mu$ y $P$ dados]
Sea $\mu$ una medida de probabilidad sobre $I$ y $P$ una matriz estocástica indexada por $I$. Entonces existe un espacio de probabilidad $(\om,\, \cal{F},\, \prob)$ y una sucesión $X = \suc{X}$ de variables aleatorias en este espacio tal que $X$ es una cadena de Markov con distribución inicial $\mu$ y matriz de transición $P$.
\end{theorem}

\begin{notation}
Se escribirá $X \dist \CMf{\mu}{P}$ si $X$ es una cadena de Markov con distribución inicial $\mu$ y matriz de transición $P$.
\end{notation}

\begin{theorem}[Construcción directa de cadenas de Markov]
Sea $\espProb$ un espacio de probabilidad en el cual están definidas variables aleatorias independientes $\suc{U}$ con distribución $\Uniff{0}{1}$ y adicionalmente una variable aleatoria $\xi_{0}$ a valores $I$ con distribución $\mu$, independiente de todas las $U_n$. Sea $\Phi:[0,\, 1] \times I \to I$ medible y definimos $\suc{X}$ mediante
\begin{align}
    X_0 &{}={} \xi_0 \notag \\
    X_{n+1} &{}={} \Phi (U_n,\, X_n), \quad n \geq 0
\end{align}
Entonces $X \dist \CMf{\mu}{P}$ con $P$ dada por $p_{ij}~=~\probf{\Phi(U_0,\, i) = j}$
\end{theorem}