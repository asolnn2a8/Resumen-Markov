%%% template.tex
%%%
%%% This LaTeX source document can be used as the basis for your technical
%%% paper or abstract. Intentionally stripped of annotation, the parameters
%%% and commands should be adjusted for your particular paper - title, 
%%% author, article DOI, etc.
%%% The accompanying ``template.annotated.tex'' provides copious annotation
%%% for the commands and parameters found in the source document. (The code
%%% is identical in ``template.tex'' and ``template.annotated.tex.'')

\documentclass[]{reporte}
\usepackage{algorithm}
\usepackage[noend]{algpseudocode}
\TOGonlineid{45678}
\TOGvolume{0}
\TOGnumber{0}
\TOGarticleDOI{0}
\TOGprojectURL{}
\TOGvideoURL{}
\TOGdataURL{}
\TOGcodeURL{}
\usepackage{color}
%\definecolor{red}{rgb}{0.9, 0.17, 0.31}
\usepackage{multirow}
\usepackage{subfig}
\usepackage{xcolor}
\usepackage{lipsum}
\usepackage{listings}
\usepackage{graphicx}
\usepackage{amsmath}
\usepackage{hyperref}
\usepackage{lipsum}
% \ProvidesPackage{mathenv}[2020/06/07 Entornos Matematicos Usuales]

\newtheorem{theorem}{Teorema}[section]
\newtheorem{corollary}[theorem]{Corolario}
\newtheorem{lemma}[theorem]{Lema}
\newtheorem{definition}[theorem]{Definición}
\newtheorem{proposition}[theorem]{Proposición}
\newtheorem{notation}[theorem]{Notación}
\usepackage{macros/mathenv} % Paquete propio para insertar entornos matemáticos
\usepackage{macros/probability} % Paquete propio para comandos de probabilidades
\usepackage{macros/mathcmd} % Paquete propio para comandos usuales de Beauchef

% Las eciaciones se enumeran por la sección
\numberwithin{equation}{section}

% Definición de la información del documento
\title{Resumen del control 1 de Procesos de Markov}
\author{}
\pdfauthor{\theauthor}
\renewcommand{\nombrecurso}{Prcesos de Markov}
\renewcommand{\codigocurso}{MA5406-1}
\renewcommand{\nombreuniversidad}{Universidad de Chile, FCFM}
% Definición de la información del documento

\begin{document}

\maketitle

%%%%%%%%%%%%%%%%% CONTENIDO %%%%%%%%%%%%%%%%%
\section{Breve recuerdo de probabilidades}

\begin{definition}[Espacio de Probabilidad]
Un \textbf{espacio de probabilidad} es un espacio de medida $(\om,\, \cal{F},\, \prob)$ tal que $\prob(\om) = 1$.
\end{definition}

\begin{definition}[Variable aleatoria]
Dado un espacio medible $(S,\, \cal{S})$, una \textbf{variable aleatoria} $X$ a valores en $S$ es una función $X:\om\to S$ que es $\cal{F}- \cal{S}$ medible.
\end{definition}

\begin{definition}[Independencia]
Dos eventos $A$ y $B$ son \textbf{independientes} si $\probf{A\cap B} = \probf{A} \probf{B}$
\end{definition}

\begin{definition}[Probabilidad condicional]
La \textbf{Probabilidad~condicional} de $A$ dado $B$ es
\begin{equation}
    \probf{A|B} = \frac{\probf{A \cap B}}{\probf{B}}
\end{equation}
\end{definition}
\section{Procesos estocásticos y cadenas\\
de Markov}
\begin{definition}[Proceso estocástico]
Dados $(\om,\, \cal{F},\, \prob)$ y $(S,\, \cal{S})$, decimos que una sucesión $(X_n)_{n\geq0}$ de variables aleatorias es un \textbf{proceso estocástico}.
\end{definition}

\begin{definition}[Espacio de estados]
el \textbf{espacio de estados} se denotará por $I$, qu es un conjunto numerable en donde los procesos estocásticos tomarán sus valores.
\end{definition}


\begin{definition}[Cadena de Markov]
Un proceso estocástico $X = \suc{X}$ a valores en $I$ es un \textbf{proceso de Markov a tiempo discreto} si satisface la \textbf{propiedad de Markov}:
\begin{multline}
    \probf{X_{n+1} = i_{n+1} | X_0 = i_0,\, \ldots,\, X_n = i_n} \\
    = \probf{X_{n+1} = i_{n+1} | X_n = i_n}
\end{multline}
\end{definition}

\begin{proposition}
Si $X$ es una cadena de Markov, entonces para todo $0 \leq t_0 \leq \ldots \leq t_{n+1}$ en $\N$ y todo $i_0,\, \ldots,\, i_{n+1} \in I$,
\begin{multline}
    \probf{X_{t_{n+1}} = i_{n+1} | X_{t_{0}} = i_0,\, \ldots,\, X_{t_{n}} = i_n} \\
    = \probf{X_{t_{n+1}} = i_{n+1} | X_{t_n} = i_n}
\end{multline}
\end{proposition}

\begin{definition}[Cadena de Markov homogénea]
Sea $X$ una cadena de Marjok. Decimos que $X$ es \textbf{homogénea} si
\begin{multline}
    \probf{X_{m+1} = j | X_m = i}\\
    = \probf{X_{n+1} = j | X_n = i} \quad \forall m, n \geq 0,\, i, j \in I
\end{multline}
\end{definition}

\begin{definition}[Matriz de transición y Prob. de transición]
De la definición anterior, la matriz $P = (p_{ij})_{i,j \in I}$ dada por $p_{ij} = \probf{X_1 = j | X_0 = i}$ le llamamos \textbf{matriz de transición} de la cadena de Markov $X$, y $p_{ij}$ es la \textbf{probabilidad de transición de $i$ a $j$}.
\end{definition}

\begin{definition}[Matriz estocástica]
una matriz $P = (p_{ij})$ se dice \textbf{matriz estocástica} si:
\begin{itemize}
    \item $p_{ij} \geq 0$ para todo $i,j \in I$.
    \item $\sum_{j\in I} p_{ij} = 1$ para todo $i\in I$.
\end{itemize}
\end{definition}

\begin{proposition}
Si $P$ es la matriz de transición de una cadena de Markov, entonces $P$ es matriz estocástica.
\end{proposition}
\section{Caracterización y construcción}

\begin{definition}[Distribución inicial]
Al vector de probabilidad $\mu = (\mu_{i})_{i\in I}$ definido por $\mu_{i\in I} = \probf{X_0 = i}$ le llamamos la \textbf{distribución inicial} de la cadena de Markov $X$.
\end{definition}

\begin{proposition}
$X$ es una cadena de Markov con matriz de transición $P$ y distribución inicial $\mu$ si y solo si
\begin{multline}
    \probf{X_0 = i_0,\, X_1 = i_1,\, \ldots,\, X_n = i_n} \\
    = \mu_{i_0} p_{i_0 i_1} \ldots p_{i_{n-1} i_n}
\end{multline}
\end{proposition}

\begin{theorem}[Existencia de una CM con $\mu$ y $P$ dados]
Sea $\mu$ una medida de probabilidad sobre $I$ y $P$ una matriz estocástica indexada por $I$. Entonces existe un espacio de probabilidad $(\om,\, \cal{F},\, \prob)$ y una sucesión $X = \suc{X}$ de variables aleatorias en este espacio tal que $X$ es una cadena de Markov con distribución inicial $\mu$ y matriz de transición $P$.
\end{theorem}

\begin{notation}
Se escribirá $X \dist \CMf{\mu}{P}$ si $X$ es una cadena de Markov con distribución inicial $\mu$ y matriz de transición $P$.
\end{notation}

\begin{theorem}[Construcción directa de cadenas de Markov]
Sea $\espProb$ un espacio de probabilidad en el cual están definidas variables aleatorias independientes $\suc{U}$ con distribución $\Uniff{0}{1}$ y adicionalmente una variable aleatoria $\xi_{0}$ a valores $I$ con distribución $\mu$, independiente de todas las $U_n$. Sea $\Phi:[0,\, 1] \times I \to I$ medible y definimos $\suc{X}$ mediante
\begin{align}
    X_0 &{}={} \xi_0 \notag \\
    X_{n+1} &{}={} \Phi (U_n,\, X_n), \quad n \geq 0
\end{align}
Entonces $X \dist \CMf{\mu}{P}$ con $P$ dada por $p_{ij}~=~\probf{\Phi(U_0,\, i) = j}$
\end{theorem}
%%%%%%%%%%%%%%%%% CONTENIDO %%%%%%%%%%%%%%%%%

\end{document}
